\section*{General Introduction}
\fancyhead[R]{General Introduction}
\addcontentsline{toc}{chapter}{General Introduction}
\begingroup
\setstretch{1.5}
In the electronic payment ecosystem, financial institutions face significant challenges processing transaction fees from international networks. \textbf{Complex referential documents} and \textbf{Dual Messaging Systems (DMS)} create substantial operational hurdles for banks, limiting fee visibility and complicating validation processes.

This project, part of the \textit{Payment Steering Solution (P2S)} initiative, introduces a \textbf{two-tier architecture} for transforming payment scheme documentation into standardized data structures. The system consists of a \textit{Document Normalization Layer} and a \textit{Transaction Processing Layer} working in concert to enhance fee calculation accuracy.

\textit{The Document Normalization Layer} processes referential documents from various payment networks through specialized extraction strategies tailored to each network's documentation format. This component produces \textbf{standardized fee rule repositories} with consistent condition attributes, metadata, and rate information.

\textit{The Transaction Processing Layer} leverages \textit{distributed message processing} with optimized data serialization to achieve high-throughput transaction matching. Our innovative \textbf{\textit{Cost Code (CC)}} system enables efficient transaction-to-rule mapping through optimized database access patterns, providing pre-clearing fee visibility while maintaining performance under high volume.

\setlength{\parindent}{0pt}
The report contains 4 main chapters:
\begin{itemize}[label={$\star$}]
\item \textbf{Chapter 1:} ``Project and Project Management''
\item \textbf{Chapter 2:} ``Analysis and Design''
\item \textbf{Chapter 3:} ``Implementation''
\item \textbf{Chapter 4:} ``Evaluation and Metrics''
\end{itemize}
\vspace{1cm}
A work summary highlighting essential contributions and future perspectives concludes the report.
\endgroup
\pagebreak