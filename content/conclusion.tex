\section*{Conclusion and Perspectives}
\fancyhead[R]{Conclusion and Perspectives}
\addcontentsline{toc}{section}{Conclusion and Perspectives}

This internship project has successfully delivered a comprehensive proof-of-concept for event-driven P2S compute engines, validating the fundamental architectural concepts for stateless fee calculation in financial transaction processing environments.

\textbf{Project Accomplishments:}

The implementation achieved its core objectives by developing a functional prototype that demonstrates the viability of event-driven, stateless processing architecture for fee calculation. The system successfully processed transactions at 1038.12 TPS, exceeding the baseline target of 1000+ TPS while maintaining 100\% accuracy and reliability.

Key technical achievements include the successful implementation of four specialized compute engines (Meta Worker, Context Qualification, Interchange Calculation, and Scheme Fee Calculation), each operating independently with efficient resource utilization. The stateless architecture pattern was validated through practical implementation, demonstrating horizontal scaling capabilities and fault tolerance.

\textbf{Technical Validation:}

The proof-of-concept successfully validated several critical architectural concepts:
\begin{itemize}
    \item \textbf{Stateless Processing:} All compute engines operate without session state, enabling linear horizontal scaling
    \item \textbf{Event-Driven Architecture:} Kafka-based messaging ensures reliable transaction processing with exactly-once semantics
    \item \textbf{Microservices Decomposition:} Independent engine deployment allows for targeted optimization and scaling
    \item \textbf{Technology Stack Validation:} The chosen combination of Java, Spring Boot, Kafka, PostgreSQL, and Redis proved effective for financial processing requirements
\end{itemize}

\textbf{Challenges and Learning:}

The project provided valuable insights into enterprise software development realities, particularly regarding data model evolution and requirements management in financial technology environments. The experience with evolving input schemas and the need for flexible architecture design demonstrated the importance of defensive programming and adaptive system design in real-world implementations.

Infrastructure constraints during development highlighted the significant performance improvements achievable through dedicated production environments, providing clear optimization pathways for future implementation phases.

\textbf{Business Impact:}

The proof-of-concept establishes a solid foundation for Effyis Group's P2S platform development, providing validated technical confidence for proceeding with full-scale production implementation. The demonstrated architecture patterns and performance baselines offer concrete metrics for client engagement and proposal development.

\textbf{Future Perspectives:}

Several enhancement opportunities have been identified for future development phases:

\begin{itemize}
    \item \textbf{Performance Optimization:} Infrastructure scaling and connection pool optimization could significantly improve throughput
    \item \textbf{Enhanced Analytics:} Additional reporting and monitoring capabilities for operational visibility
    \item \textbf{Extended Fee Scenarios:} Support for additional payment network fee structures and business rules
    \item \textbf{Production Deployment:} Implementation of the designed multi-zone deployment architecture
    \item \textbf{Operational Monitoring:} Enhanced system monitoring and alerting capabilities for production environments
\end{itemize}

\textbf{Professional Development:}

This internship provided extensive exposure to enterprise financial technology development, event-driven architecture patterns, and modern microservices implementation. The experience with performance testing, architectural design, and stakeholder management in a complex technical environment has been invaluable for professional growth.

The project successfully demonstrates the transition from theoretical computer science concepts to practical, production-oriented software engineering, highlighting the importance of balancing technical excellence with business requirements and implementation constraints.

In conclusion, the proof-of-concept has fulfilled its objectives by establishing a validated technical foundation for P2S compute engine development while providing clear pathways for production-scale implementation and optimization.

\pagebreak