\section{Work Technologies}
\textbf{Java:}
Java was the primary programming language for the development of this project. Java was chosen for its robustness, portability, and extensive ecosystem, which make it suitable for building scalable and maintainable applications. Its strong type system and mature libraries contributed to reliable code, while its widespread adoption ensured access to a large pool of resources and community support.
\begin{figure}[H]
\centering
\includegraphics[width=0.2\textwidth]{img/tech/java-logo.png}
\caption{Java logo}
\end{figure}

\textbf{Spring Boot:}
Spring Boot is a framework that simplifies the development of Java applications by providing auto-configuration and embedded servers. It enables rapid development of production-ready applications with minimal configuration, while offering comprehensive features for dependency injection, security, and data access.
\begin{figure}[H]
\centering
\includegraphics[width=0.3\textwidth]{img/tech/springboot-logo.png}
\caption{Spring Boot logo}
\end{figure}

\textbf{PostgreSQL:}
PostgreSQL is an advanced, open-source relational database management system known for its reliability, feature robustness, and performance. It was used as the primary database for this project, providing ACID compliance, advanced indexing, and support for complex queries and data types.
\begin{figure}[H]
\centering
\includegraphics[width=0.2\textwidth]{img/tech/postgres-logo.png}
\caption{PostgreSQL logo}
\end{figure}

\textbf{Redis:}
Redis is an in-memory data structure store used as a database, cache, and message broker. In this project, Redis was utilized for caching frequently accessed data and session management, significantly improving application performance and reducing database load.
\begin{figure}[H]
\centering
\includegraphics[width=0.2\textwidth]{img/tech/redis-logo.png}
\caption{Redis logo}
\end{figure}

\textbf{Docker:}
Docker is a platform for developing, shipping, and running applications in containers. It ensures that the application runs consistently across different environments, simplifying deployment and scaling while isolating dependencies.
\begin{figure}[H]
\centering
\includegraphics[width=0.3\textwidth]{img/tech/docker-logo.png}
\caption{Docker logo}
\end{figure}

\textbf{GitLab:}
GitLab is a software development platform that allows developers to collaborate, manage version control, and store their projects using Git, a distributed version control system.
\begin{figure}[H]
\centering
\includegraphics[width=0.3\textwidth]{img/logos/gitlab-logo.png}
\caption{GitLab logo}
\end{figure}

\textbf{IntelliJ IDEA:}
IntelliJ IDEA is a powerful integrated development environment (IDE) for Java development. It was used throughout the project for code development, debugging, and refactoring, providing intelligent code completion, advanced debugging tools, and seamless integration with build tools and version control systems.
\begin{figure}[H]
\centering
\includegraphics[width=0.3\textwidth]{img/tech/IntelliJ IDEA.png}
\caption{IntelliJ IDEA logo}
\end{figure}

\textbf{Excalidraw:}
Excalidraw is a virtual whiteboard tool for creating hand-drawn-like diagrams and sketches. It was used in this project for creating wireframes, system architecture diagrams, and visual representations of workflows, facilitating better communication and planning among team members.
\begin{figure}[H]
\centering
\includegraphics[width=0.3\textwidth]{img/logos/excalidraw-logo.png}
\caption{Excalidraw logo}
\end{figure}

\textbf{Apache Kafka:}
Apache Kafka is a distributed event streaming platform used for building real-time data pipelines and streaming applications. It allows for the efficient handling of large volumes of data in real-time, making it suitable for applications that require high throughput and low latency.
\begin{figure}[H]
\centering
\includegraphics[width=0.2\textwidth]{img/tech/kafka-logo.png}
\caption{Apache Kafka logo}
\end{figure}

\textbf{Apache Avro:}
Apache Avro is a data serialization framework that provides compact, fast, binary data format with rich data structures. It was used in this project for schema evolution and data serialization in Kafka messages, ensuring efficient data transmission and backward compatibility.
\begin{figure}[H]
\centering
\includegraphics[width=0.2\textwidth]{img/tech/apache-avro-logo.png}
\caption{Apache Avro logo}
\end{figure}

\textbf{Confluent Schema Registry:}
Confluent Schema Registry is a centralized repository for managing and validating schemas for topic message data. It works seamlessly with Apache Avro to provide schema evolution capabilities and ensures data compatibility across different versions of applications consuming Kafka topics.
\begin{figure}[H]
\centering
\includegraphics[width=0.3\textwidth]{img/tech/CFLT_BIG.png}
\caption{Confluent Schema Registry logo}
\end{figure}