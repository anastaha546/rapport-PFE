\section{Methodology}
\subsection{Waterfall model}
The Waterfall Model illustrates the software development process in a linear sequential flow. This means that any phase in the development process begins only if the previous phase is complete. In this waterfall model, the phases do not overlap.
\\
The Waterfall method is particularly used today for projects where the completion of each stage depends on the previous stage, for small-sized projects, and when the domain is well understood by the team.\\
\begin{figure}[H]
\centering
  \includegraphics[width=0.8\textwidth]{img/waterfall_model.png}
\caption{Waterfall model}
\label{Waterfall model}      
\end{figure}
Before moving to the next phase in the waterfall process, there's usually a review and sign off to ensure all defined goals have been met. For example, developers would ensure each unit of technology is properly integrated in the implementation phase before moving to the testing phase.\\
\subsection{Why waterfall model ?}
The Waterfall methodology was chosen for the project due to its sequential and linear approach, clear requirements, effective resource and time management, alignment with the course curriculum, and team familiarity. It provided a structured framework for the project's progression and ensured efficient utilization of resources.\\
